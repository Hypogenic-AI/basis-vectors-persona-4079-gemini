\section{Methodology}
\label{sec:methodology}

We investigate the global geometry of persona representations by performing Principal Component Analysis (	extsc{pca}) on activation-level persona vectors extracted from a diverse set of behavioral datasets. 
Our methodology follows a three-stage approach: (1) activation extraction, (2) basis discovery via 	extsc{pca}, and (3) causal validation via activation steering.

\subsection{Activation Extraction and Data Construction}
We utilize the \caa datasets \cite{rimsky2023steering}, which provide contrastive pairs of questions and answers for seven distinct behaviors: \sycophancy, \survival, \coordination, \hallucination, \corrigibility, \myopic, and 
efusal. 
For each behavior, we sample $N=100$ examples, resulting in a total of $N_{total}=700$ samples. 
Each sample consists of a question $q$, a matching answer $a_{pos}$, and a non-matching answer $a_{neg}$.

For each sample $i$, we extract the residual stream activations $h(x, l)$ at layer $l$ and the last token position of the concatenated string $x$. 
We define the persona vector $v_{i,l}$ as the difference between the positive and negative activations:
\begin{equation}
    v_{i,l} = h(q + a_{pos}, l) - h(q + a_{neg}, l)
\end{equation}
This difference vector represents the specific behavioral direction for sample $i$ at layer $l$. 
We use \qwen (a 28-layer transformer) as our base model.

\subsection{Basis Discovery via PCA}
To find the primary components of the persona space, we aggregate the difference vectors $v_{i,l}$ across all seven behaviors at a given layer $l$. 
This results in a matrix $V_l \in \mathbb{R}^{N_{total} 	imes d}$, where $d$ is the residual stream dimension ($d=1536$ for \qwen). 
We perform 	extsc{pca} on the mean-centered matrix $\bar{V}_l$ to obtain the Principal Components (PCs) and their corresponding Explained Variance Ratios (\evr). 

The \evr for the $k$-th component is given by:
\begin{equation}
    	extsc{evr}_k = \frac{\lambda_k}{\sum_{j=1}^d \lambda_j}
\end{equation}
where $\lambda_k$ is the $k$-th eigenvalue of the covariance matrix of $V_l$. 
This allows us to quantify the low-rank nature of the persona space across different layers. 
We also compute the cosine similarity between the behavioral mean vectors and the shared PCs to interpret the semantic meaning of each basis vector.

\subsection{Causal Validation via Steering}
To verify that the discovered PCs are causally meaningful, we use the top basis vector (\pcone) as a steering vector in an unseen behavioral task (
efusal). 
During the model's forward pass, we add the steering vector to the residual stream activations at Layer 14:
\begin{equation}
    h'(x, l) = h(x, l) + \alpha \cdot \hat{v}_{PC1}
\end{equation}
where $\hat{v}_{PC1}$ is the normalized PC1 vector and $\alpha$ is a steering coefficient. 
We measure the change in the probability of a refusal response as a function of the steering coefficient to evaluate the causal influence of the primary basis vector. 
This allows us to test if the identified ``Social-Self'' axis can broadly influence behavioral outputs beyond the specific traits used for its discovery.
