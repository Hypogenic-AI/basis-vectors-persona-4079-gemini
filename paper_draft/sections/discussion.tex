\section{Discussion}
\label{sec:discussion}

The discovery of a highly low-rank structure in persona space, with a single basis vector (\pcone) explaining 60\% of the variance in middle layers, has significant implications for our understanding of LLM behaviors. 
In this section, we interpret the ``Social-Self'' axis, discuss the layer-wise dissipation of this structure, and acknowledge the limitations of our study.

\para{Interpretation of the ``Social-Self'' Axis.} 
The alignment of \pcone with \sycophancy, \survival, and \coordination reveals a shared underlying axis in \qwen's residual stream. 
This cluster of social compliance and self-preservation traits suggests that these behaviors are not independent, but rather composed of a common geometric direction. 
This finding provides a mechanistic explanation for the clustering of related behaviors, such as the observed link between sycophancy and social compliance. 
The ability to steer multiple related traits through a single, foundational direction could significantly simplify model monitoring and behavioral control.

\para{Layer-wise Specialization versus Unification.} 
The dissipation of the low-rank structure across layers provides insight into how persona representations are transformed as they progress through the model. 
Our analysis showed that in early-middle layers (Layers 14--18), the model appears to represent behaviors in a highly unified, low-dimensional subspace. 
As the activations move toward the final layers, the variance concentration decreases, indicating a transition from a unified behavioral axis to more specialized, distributed representations closer to the output head. 
This suggests that ``persona'' starts as a unified behavioral direction and becomes more specialized or distributed closer to the output head, consistent with findings by Cintas et al. \cite{cintas2025localizing} that personas are more divergent in the final third of layers.

\para{Limitations and Future Work.} 
Our study is subject to several limitations. 
First, we focused on a single model architecture (\qwen) and a limited set of behavioral datasets. 
While our findings are consistent with prior work on other architectures, it remains to be seen if the identified ``Social-Self'' axis is universal across different model families (e.g., Llama, Mistral, Gemma). 
Second, our analysis of steering results was limited to a single task (
efusal) and showed a non-linear relationship between steering coefficients and behavioral probabilities. 
Future work should investigate the effects of steering on a wider range of tasks and explore more complex steering methods, such as those that combine multiple principal components. 
Furthermore, it would be valuable to investigate if PC2 and PC3 represent other foundational behavioral axes, such as ``Honesty'' or ``Competence'', and how these dimensions interact with each other in the model's residual stream.
Finally, we aim to explore if it is possible to specifically ``zero out'' the sycophancy component of PC1 without affecting other related behaviors, potentially leading to more targeted and efficient model alignment.
