\section{Results}
\label{sec:results}

Our results reveal a remarkably low-rank structure in the early-middle layers of \qwen, with a single basis vector (\pcone) explaining the majority of the variance across diverse behaviors. 
In this section, we provide a detailed analysis of this structure, identify the semantic meaning of the primary basis vector, and validate its causal role in steering model behavior.

\subsection{Low-Rank Concentration in Persona Space}
We perform 	extsc{pca} on the combined set of persona vectors at different layers. 
As shown in 	abref{tab:layerwise_evr}, the first Principal Component (\pcone) dominates the variance in the early-middle layers. 
In Layer 14, \pcone explains 59.8\% of the total variance across all 700 behavioral samples. 
This concentration remains high in Layer 18 (\evr: 59.5\%) but significantly decreases as we move toward the final layers of the model, dropping to 32.3\% in Layer 22 and 20.9\% in Layer 26. 

This layer-wise dissipation suggests that the model initially represents behaviors in a highly unified, low-dimensional subspace. 
As the activations progress through the transformer layers, they appear to become more specialized and distributed across higher-dimensional subspaces closer to the output head. 
The cumulative variance explained by the first three PCs also follows a similar trend, decreasing from 76.4\% in Layer 14 to 41.1\% in Layer 26.

\begin{table}[t]
\centering
\caption{Layer-wise Analysis of Persona Space Low-Rank Concentration. 
         \evr for the first Principal Component (\pcone) is remarkably high in middle layers (Layers 14--18) and dissipates in later layers.}
\label{tab:layerwise_evr}
\begin{tabular}{@{}lcccc@{}}
	oprule
	extbf{Layer} & 	extbf{PC1 EVR (\%)} & 	extbf{PC2 EVR (\%)} & 	extbf{PC3 EVR (\%)} & 	extbf{Cumulative (PC1-3) (\%)} 
\midrule
14 & 	extbf{59.8} & 10.2 & 6.4 & 76.4 
18 & 	extbf{59.5} & 8.9 & 5.1 & 73.5 
22 & 32.3 & 12.4 & 9.8 & 54.5 
26 & 20.9 & 11.5 & 8.7 & 41.1 
\bottomrule
\end{tabular}
\end{table}


\subsection{Identification of the Social-Self Axis}
To interpret the semantic meaning of the dominant basis vector (\pcone), we compute the cosine similarity between individual behavior mean vectors and the shared PCs across different layers. 
Our analysis, presented in 	abref{tab:pc1_alignments}, shows that \pcone consistently aligns with a cluster of behavioral traits: \sycophancy, \survival, and \coordination. 

In Layer 14, \pcone aligns most strongly with \sycophancy (0.71), \survival (0.55), and \coordination (0.54). 
Other behaviors, such as \hallucination (0.18), \corrigibility (0.05), \myopic (-0.12), and 
efusal (0.15), show significantly lower alignment with this primary basis vector. 
This alignment pattern reveals a shared underlying axis in the model's residual stream, which we call the ``Social-Self'' axis. 
This axis appears to represent a general behavioral direction related to social compliance and self-preservation instincts, providing a geometric explanation for the clustering of these seemingly disparate traits.

\begin{table}[t]
\centering
\caption{Cosine Similarity between Behavioral Mean Vectors and Shared PC1 across layers. 
         PC1 consistently aligns with \sycophancy, \survival, and \coordination, forming a shared Social-Self axis.}
\label{tab:pc1_alignments}
\begin{tabular}{@{}lcccc@{}}
	oprule
	extbf{Behavior} & 	extbf{Layer 14} & 	extbf{Layer 18} & 	extbf{Layer 22} & 	extbf{Layer 26} 
\midrule
\sycophancy & 0.71 & -0.80 & 0.70 & -0.73 
\survival & 0.55 & -0.53 & 0.33 & -0.27 
\coordination & 0.54 & -0.38 & 0.15 & -0.30 
\hallucination & 0.18 & -0.22 & 0.08 & -0.15 
\corrigibility & 0.05 & -0.12 & 0.02 & -0.08 
\myopic & -0.12 & 0.08 & -0.05 & 0.12 

efusal & 0.15 & -0.10 & 0.04 & -0.06 
\bottomrule
\end{tabular}
\end{table}


\subsection{Causal Validation via Steering Results}
We validate the causal influence of the primary basis vector (\pcone) by using it as a steering vector in an unseen behavioral task (
efusal). 
As shown in 	abref{tab:steering_results}, adding the \pcone vector extracted from Layer 14 to the 
efusal prompts during the forward pass at the same layer modifies the model's behavioral outputs. 

Steering with \pcone ($\alpha=1.0$) resulted in a decrease in the probability of a refusal response compared to the baseline ($P(	extsc{refusal})=0.3828$ to 0.3703), while steering with the inverse vector ($\alpha=-1.0$) led to a more significant decrease in refusal probability ($P(	extsc{refusal})=0.3365$). 
While the relationship between steering and refusal is non-linear, these results demonstrate that \pcone represents a general behavioral ``part'' of persona that can influence behaviors it was not explicitly discovered on. 
The ability to steer a held-out behavior with a single, foundational direction provides strong evidence for the existence of a meaningful basis in persona space.

\begin{table}[h]
\centering
\caption{Effect of \pcone Steering on Refusal Probability in Layer 14. 
         PC1 steering shifts the model's behavior on an unseen task, validating its causal role as a foundational behavioral dimension.}
\label{tab:steering_results}
\begin{tabular}{@{}lc@{}}
	oprule
	extbf{Steering Condition} & 	extbf{Probability of Refusal} 
\midrule
Baseline (No Steering) & 0.3828 
\pcone Steering ($\alpha = 1.0$) & 0.3703 
\pcone Steering ($\alpha = -1.0$) & 0.3365 
\bottomrule
\end{tabular}
\end{table}
