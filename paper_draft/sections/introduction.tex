\section{Introduction}
\label{sec:introduction}

Large Language Models (LLMs) often exhibit diverse and complex behaviors, ranging from human-like personas and political views to subtle biases like sycophancy or self-preservation instincts. 
As these models are increasingly used in interactive and safety-critical applications, the ability to interpret and steer these internal representations is of paramount importance. 
While recent work has localized where these personas are encoded \cite{cintas2025localizing} and proposed methods for steering them via activation-level interventions \cite{rimsky2023steering, wang2025geometry}, the global geometric structure of the ``persona space'' across multiple, seemingly disparate behavioral traits remains poorly understood. 
{\bf Does a common underlying basis exist that explains the relationship between different personas in the model's residual stream?}

Existing research typically examines persona traits in isolation, such as focusing on the Big Five personality traits or specific ethical biases. 
However, this narrow focus overlooks the possibility that many complex behaviors may be composed of a small set of primary basis vectors in the high-dimensional activation space. 
If such a basis exists, it would significantly simplify the task of model monitoring and behavioral control, allowing for the steering of multiple related traits through a single, foundational direction. 
Furthermore, identifying these basis vectors could reveal the fundamental axes of LLM behavior, providing a mechanistic explanation for why models exhibit clusters of related traits, such as the observed link between sycophancy and social compliance.

In this work, we propose \ours to investigate the global geometry of persona representations in \qwen. 
We perform Principal Component Analysis (	extsc{pca}) on activation-level persona vectors extracted from seven diverse behavioral domains: \sycophancy, \survival, \coordination, \hallucination, \corrigibility, \myopic, and 
efusal. 
By analyzing the residual stream activations at the last token position across multiple layers, we identify the primary components that explain the variance across these behaviors. 
Our analysis reveals a remarkably low-rank structure in the middle layers of the model, where a single basis vector (\pcone) dominates the representation of social and self-preservation traits. 

Our results indicate that in Layer 14 of the 28-layer \qwen model, \pcone explains 59.8\% of the total variance across all seven behavioral datasets. 
This ``Social-Self'' axis is strongly aligned with \sycophancy, \survival, and \coordination, suggesting a unified behavioral direction for these traits. 
We validate the causal influence of this basis vector by using it as a steering vector in an unseen 
efusal task, demonstrating its ability to modify the model's behavioral probabilities. 
Finally, we show that this low-rank concentration dissipates in later layers, with \evr decreasing from 60\% in Layer 14 to 20.9\% in Layer 26, indicating a transition from a unified behavioral axis to more specialized, distributed representations.

Our main contributions are as follows:
\begin{itemize}[leftmargin=*,itemsep=0pt,topsep=0pt]
    \item We identify a highly low-rank structure in the early-middle layers of the residual stream, where a single basis vector explains nearly 60\% of the variance across diverse behaviors.
    \item We discover a shared ``Social-Self'' axis that unifiedly represents \sycophancy, \survival, and \coordination, providing a new geometric lens for understanding these traits.
    \item We characterize the layer-wise dissipation of persona representations, showing that unified behavioral directions break down into more specialized components closer to the output head.
    \item We provide causal validation for the identified basis vector by demonstrating its efficacy as a steering vector on a held-out behavioral task.
\end{itemize}
